\documentclass[11pt, oneside, doublespacing]{article}   	% use "amsart" instead of "article" for AMSLaTeX format
\usepackage{geometry}                		% See geometry.pdf to learn the layout options. There are lots.
\geometry{letterpaper}                   		% ... or a4paper or a5paper or ... 
%\geometry{landscape}                		% Activate for rotated page geometry
%\usepackage[parfill]{parskip}    		% Activate to begin paragraphs with an empty line rather than an indent
\usepackage{graphicx}				% Use pdf, png, jpg, or eps§ with pdflatex; use eps in DVI mode
								% TeX will automatically convert eps --> pdf in pdflatex		

\usepackage{amsmath, amsthm, amssymb}
\usepackage{setspace}
\doublespacing
%SetFonts

%SetFonts


\title{HW2}
\author{Azalee Bostroem}
%\date{}							% Activate to display a given date or no date

\begin{document}
\maketitle
\section*{Putting everything into energy units}
In SI units: \\
$\hbar = 1.054x10^{-34}$ (J s) \\
$k_B = 1.381x10^{-23} {\rm\left(\frac{J}{K}\right)} $
$c = 2.99x10^{8} {\rm \frac{m}{s}}$ \\
$G = 6.67x10^{-11} {\rm \frac{m^3}{(kg)(s^2)}}$ \\
$H_{0} = 65 {\rm \frac{km}{(Mpc)(s)}} \times \frac{1000m}{1km} \times\frac{1Mpc}{3.08x10^{22}m} = 2.11x10^{-18}{\rm \frac{1}{s}}$ \\

\noindent  In energy units c = ${\rm \hbar}$ = ${\rm k_{B}}$ = 1. To convert other constants to energy units: \\
$G = 6.67x10^{-11} {\rm \frac{m^3}{(kg)(s^2)}}\times \frac{1}{(\hbar c)^3} \times \frac{1}{c^2} \times \frac{\hbar ^2}{1} = 6.67x10^{-11} \times \frac{1}{1.054x10^{-34} * (2.99x10^{8})^5} = 2.61\times10^{-19}{\rm{\frac{1}{J^{2}}}}$  \\
$H_{0} = 2.11x10^{-18}{\rm \frac{1}{s}} \times \frac{\hbar}{1} = 2.11x10^{-18} * 1.054x10^{-34}  = 2.22 \times 10^{-52} $J  \\
$T_{0} = 2.27 {\rm K} \times k_{B} = 2.27 * 1.381x10^{-23} = 3.13 \times 10^{-23} $ J \\

%%%%%%%%%%%%%%%%%%%%%%
\section*{Problem 2.1}
\noindent As derived in problem 2.5 
\begin{equation}
\rho = \rho_{0} \left( \frac{a}{a_{0}}\right) ^{-3(1+w)}
\label{eqn:Density}
\end{equation}
Given: \\
$w_m = 0$ \\
$w_r = \frac{1}{3}$ \\
$w_{\Lambda} = -1$ \\
\\
And defining the following values at the present time: \\
$a_{0} = 1$ \\
$\rho_{0, c} = \frac{3*c^2}{8 \pi G}H_{0}^2 = \frac{3}{8 \pi 2.61x10^{-19}} * (2.22 x 10^{-52})^{2} = 2.25x10^{-86}{\rm J^{4}}$  \\
$\rho_{0, m} = 0.28 \rho_{0, c} = 6.31x10^{-87}{\rm J^{4}}$ \\
$\rho_{0, \Lambda} = 0.72 \rho_{0, c} = 1.62x10^{-86}{\rm J^{4}}$ \\
$\rho_{0, r} = \frac{\pi^2}{30} g_{*} T_{0}^4 = \\
\hspace*{17pt}= \frac{\pi^2}{30} (2) T_{0}^4 \\
\hspace*{17pt}= \frac{\pi^2}{30} (2) (3.13 x 10^{-23})^4 = 1.27x10^{-90}{\rm J^{4}}$\\
where a photon has 2 internal degrees of freedom and current photons are at \\
$T_{0}=2.7{\rm K} = 3.13 \times 10^{-23} $ J. \\

\noindent From equation \ref{eqn:Density}: \\
\begin{equation}
\rho_{m}(a) =  \frac{0.28 \rho_{0, c}}{ a^{3}}
\label{eqn:DensityM}
\end{equation}
\begin{equation}
\rho_{r}(a) = \frac{(\pi^2)(T_{0}^4)}{15 (a^{4})}  
\label{eqn:DensityR}
\end{equation}
\begin{equation}
\rho_{\Lambda}(a) = 0.72 \rho_{0, c} (a^{0}) = 0.72 \rho_{0, c}
\label{eqn:DensityL}
\end{equation}

%%%%%%%%%%%%%%%%%%%%%%
\section*{Problem 2.2}
\includegraphics[width=0.9\textwidth]{problem2_2.pdf}

%%%%%%%%%%%%%%%%%%%%%%
\newpage
\section*{Problem 2.3}
\noindent Find $a_{eq}$: \\
$\rho_{m}(a_{eq}) = \rho_{r}(a_{eq})$ \\
$\frac{0.28 \rho_{0, c}}{a_{eq}^{3}} = \frac{\pi^{2} T_{0}^{4} }{15(a_{eq})^{4}}$ \\
$a_{eq} = \frac{\pi^{2} (3.13 x 10^{-23})^{4}}{15 * 0.28 * 2.25x10^{-86}} = 2.01\times10^{-4}$  \\


\noindent Find $a_{\Lambda}$:\\
$\rho_{m}(a_{\Lambda}) = \rho_{\Lambda}(a_{\Lambda})$ \\
$\frac{0.28 \rho_{0, c}}{a_{\Lambda}^{3}} = 0.72 \rho_{0, c} $ \\
$a_{\Lambda} = \sqrt[3]{\frac{0.28}{0.72}} = 0.73 $ \\

%%%%%%%%%%%%%%%%%%%%%%

\section*{Problem 2.4}
\noindent From the First Friedman equation, with c = 1:
\begin{equation}
H^{2} = \frac{8\pi G \rho}{3} + \frac{\Lambda}{3} - \frac{k}{a^2}
\label{eqn:friedman}
\end{equation}
For a flat universe $k=0$. \\
$\frac{\Lambda}{3} \equiv \frac{8\pi G}{3}\rho_{\Lambda}$\\
$\rho = \rho_m + \rho_r $\\
Filling in these values, equation \ref{eqn:friedman} becomes
\begin{equation*}
H^{2} = \frac{8\pi G}{3} (\rho_m + \rho_r + \rho_{\Lambda})
\end{equation*}
Therefore, $\rho_{c} = \frac{3}{8\pi G}H^{2} = \rho_m + \rho_r + \rho_{\Lambda}$
Plotting:
\begin{equation}
\Omega_{i} = \frac{\rho_i}{\rho_c} = \frac{\rho_i}{\rho_m + \rho_r + \rho_{\Lambda}}
\end{equation}

\includegraphics[width=0.9\textwidth]{problem2_4.pdf}

\section*{Problem 2.5}
\noindent From FRW handout equation 2.9: \\
$\frac{d\rho}{\rho} = -3(1+w)\frac{da}{a}$ \\
$\int_{\rho_{0}}^{\rho} \frac{d\rho}{\rho} = -3(1+w)\int_{a_{0}}^{a} \frac{da}{a}$ \\
${\rm ln}(\rho) - {\rm ln}(\rho_{0}) = -3(1+w)[{\rm ln}(a) - {\rm ln}(a_{0})]$ \\
${\rm ln}\left(\frac{\rho}{\rho_0}\right) = -3(1+w){\rm ln}\left(\frac{a}{a_0}\right)$ \\
$\frac{\rho}{\rho_0} = \left(\frac{a}{a_0}\right)^{-3(1+w)} $ \\
$\rho = \rho_{0} \left( \frac{a}{a_{0}}\right) ^{-3(1+w)} $ \\


\end{document}  